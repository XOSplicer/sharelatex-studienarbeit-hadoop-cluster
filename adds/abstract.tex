% !TEX root = ../master.tex
\chapter*{Abstract}

\begingroup
  \begin{table}[h!]
    \setlength\tabcolsep{0pt}
    \begin{tabular}{p{3.5cm}p{10.0cm}}
      Title & \dertitel \\
      Author: & \derautor \\
    \end{tabular}
  \end{table}
\endgroup

\hspace{2cm}

The ever-increasing amount of data available in the current digital age 
poses major challenges for corporations, 
a phenomenon widely known as \emph{Big Data}. 
In order to gain insight from this data,
one needs to have information systems in place which 
provide a platform to analyze it in an efficient manner.

The computing power of single computers is limited by both physical 
as well as economic constraints,
hence a multitude of machines can be connected to form \emph{computing clusters}
that are capable o.

As the computing power of each computer is limited by both physical 
as well as economic constraints, 
the scale of nowadays' Big Data needs to be handled by
a multitude of connected machines that therefore form \emph{computing clusters}.
Such cluster environments can scale with the increasing amount of data managed by it. 
The various nodes of each cluster have to be organized,
e.g. by assigning tasks to a particular machine or sending data to places where it is needed. 
This work is performed by so-called \emph{cluster managers}.

A popular cluster manager for Big Data clusters in the industry is the \emph{Hadoop} framework. 
Hadoop offers distributed storage for arbitrary-sized collections of data 
by combining the individual computing and storage resources of commodity hardware. 
Furthermore, it provides programming interfaces for interacting with and analyzing the stored data.

This research project is embedded into a larger body of investigation 
regarding the aptitude of Hadoop as \acs{BIA} information system 
for use at the \acf{DHBW} in both research and administration. 
Thus, this work specifically deals with the following topics:

\begin{itemize}
    \item Examination of existing infrastructure
    \item Deployment of an Hadoop cluster on-top of it. 
    This also includes the exploration of possibilities for automatic configuration 
    and installation of such a cluster.
\end{itemize}

As an outcome of this project, 
a detailed execution plan to perform said deployment is given.
Furthermore the deployment is rolled out to the given environment. 

%TODO: Update after project


