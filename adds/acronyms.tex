% !TEX root = ../master.tex
\clearpage
\chapter*{List of Acronyms}
\addcontentsline{toc}{chapter}{List of Acronyms}

%Verwendung:
%		\ac{Abk.}   --> fügt die Abkürzung ein, beim ersten Aufruf wird zusätzlich automatisch die ausgeschriebene Version davor eingefügt bzw. in einer Fußnote (hierfür muss in header.tex \usepackage[printonlyused,footnote]{acronym} stehen) dargestellt
%		\acs{Abk.}   -->  fügt die Abkürzung ein
%		\acf{Abk.}   --> fügt die Abkürzung UND die Erklärung ein
%		\acl{Abk.}   --> fügt nur die Erklärung ein
%		\acp{Abk.}  --> gibt Plural aus (angefügtes 's'); das zusätzliche 'p' funktioniert auch bei obigen Befehlen
%	siehe auch: http://golatex.de/wiki/%5Cacronym

\begin{acronym}
	\acro{API}{Application Programming Interface}
	\acro{BDA}{Big Data Analytics}
	\acro{BIA}[BI\&A]{Business Intelligence \& Analytics}
	\acro{CDH}{Cloudera Distribution Including Apache Hadoop}
	\acro{CPU}{Central Processing Unit}
	\acro{DBMS}{Database Management System}
	\acro{DHBW}{Baden-Württemberg Cooperative State Univerity (Duale Hochschule Baden-Württemberg)}
	\acro{DM}{Data Mining}
	\acro{DNS}{Domain Name System}
	\acro{ECC}{Error-Correcting Code}
	\acro{ETL}{Extract Transform Load}
	\acro{FQDN}{Fully Qualified Domain Name}
	\acro{GB}{gigabyte}
	\acro{TB}{terabyte}
	\acro{HDFS}{Hadoop Distributed File System}
	\acro{HDP}{Hortonworks Data Platform}
	\acro{IoT}{Internet of Things}
	\acro{IP}{Internet Protocol}
	\acro{IT}{Information Technology}
	\acro{JDK}{Java Development Kit}
	\acro{LTS}{Long Term Support}
	\acro{NCDC}{National Climatic Data Center}
	\acro{NTP}{Network Time Protocol}
	\acro{OLAP}{Online Analytical Processing}
	\acro{OLTP}{Online Transaction Processing}
	\acro{VM}{Virtual Machine}
	\acro{RAM}{Random Access Memory}
    \acro{RDBMS}{Relational Database Management System}
	\acro{SSH}{Secure Shell}
	\acro{THP}{Transparent Hugh Pages}
	\acro{YARN}{Yet Another Resource Negotiator}
\end{acronym}





