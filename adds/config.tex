% !TEX root = ../master.tex

%-------------------
% 		HYPERREF
%-------------------

\usepackage[hidelinks=true]{hyperref}
 % hidelinks=true verhindert rote Ränder bei Links im Dokument.

% Zwei eigene Befehle zum Setzen von Autor und Titel. Ausserdem werden die PDF-Informationen richtig gesetzt.
\newcommand{\titel}[1]{\def\dertitel{#1}\hypersetup{pdftitle={#1}}}
\newcommand{\autor}[1]{\def\derautor{#1}\hypersetup{pdfauthor={#1}}}

%-----------------------------------
%		SCHRIFT UND ENCODING
%-----------------------------------
\usepackage[T1]{fontenc}
\usepackage[utf8]{inputenc}

\usepackage{setspace}
%\onehalfspacing
%TODO use this spacing

%---------------------------
%		BERECHNUNGEN
%---------------------------
\usepackage{calc} % Used for extra space below footsepline

%---------------------------------
%		SPRACHEINSTELLUNGEN
%---------------------------------
% Voreinstellungen für Deutsch und Englisch. Die nicht verwendete Sprache ist auszukommentieren.
% DEUTSCH
%\usepackage[ngerman]{babel}
%\usepackage[german=quotes]{csquotes}

%ENGLISCH
\usepackage[english]{babel}
\usepackage{csquotes} % Richtiges Setzen der Anführungszeichen mit \enquote{}

%----------------------------
%		BIBLIOGRAFIE
%----------------------------
% Voreinstellungen für Fußnotenzitate (Autor-Jahr), IEEE-Standard, Alphabetic-Stil und Havard-Stil. Die nicht verwendeten Stile müssen auskommentiert werden

% \usepackage[backend=biber, autocite=footnote, style=authoryear, dashed=false]{biblatex}		% Fußnotenzitate
% \usepackage[backend=biber, autocite=inline, style=ieee]{biblatex}							% IEEE-Stil
% \usepackage[backend=biber, autocite=inline, style=alphabetic]{biblatex}					% Alphabetic-Stil
% \usepackage[backend=biber, autocite=inline, style=authoryear, dashed=false]{biblatex}		% Harvard-Stil
\usepackage[backend=biber, autocite=inline, style=apa]{biblatex}							% APA-Stil

% Fußnotenzitate mit YYYY-MM-DD in Bibliographie
% \usepackage[backend=biber, autocite=footnote, style=authoryear, dashed=false, urldate=edtf, date=edtf, seconds=true]{biblatex}

% Zum Zählen der Fußnoten über Kapitel hinaus
\usepackage{chngcntr}
\counterwithout{footnote}{chapter}

\DefineBibliographyStrings{ngerman}{  %Change u.a. to et al. (german only!)
	andothers = {{et\,al\adddot}},
}

\setlength{\bibparsep}{\parskip}		%add some space between biblatex entries in the bibliography
\addbibresource{adds/bibliography.bib}	%Add file bibliography.bib as biblatex resource

%----------------------
%		ACRONYME
%----------------------
%%%
%%% WICHTIG: Installieren Sie das neueste Acronyms-Paket!!!
%%%
\makeatletter
\usepackage[printonlyused]{acronym}
\@ifpackagelater{acronym}{2015/03/20}
  {%
    \renewcommand*{\aclabelfont}[1]{\textbf{\textsf{\acsfont{#1}}}}
  }%
  {%
  }%
\makeatother

%--------------------
%		GLOSSAR
%--------------------
%\usepackage[toc]{glossaries}					% für Seitenreferenzen im Glossar
\usepackage[toc, nonumberlist]{glossaries}		% ohne Seitenreferenzen im Glossar

%---------------------
%		LISTINGS
%---------------------
\usepackage{listings}
% Listings formatieren
%\renewcommand{\lstlistlistingname}{Quelltextverzeichnis}
\lstset{numbers=left,
	numberstyle=\tiny,
	captionpos=b,
	breaklines=true,
	basicstyle=\linespread{0.8}\ttfamily\small}

%-------------------------------
%		ZUSÄTZLICHE PAKETE
%-------------------------------
\usepackage{lipsum}				% Blindtext
\setlipsumdefault{1-4}
\usepackage[pdftex]{graphicx} 			% verschiene Bildformate einbinden
\usepackage{pdfpages}		% PDF einbinden
\usepackage{varioref} 	% schönere Referenzen über \vref{}
\usepackage{caption}			% schönere Überschriften
\usepackage{booktabs}			% bessere Tabs
\usepackage{array}
\newcolumntype{P}[1]{>{\raggedright\arraybackslash}p{#1}}

%--------------------------
%		Tikz diagram library
%--------------------------
\usepackage{tikz}
\usetikzlibrary{arrows,decorations.pathmorphing,backgrounds,fit,positioning,shapes.symbols,chains}

%--------------------------
%		TpX used packages
%--------------------------
\usepackage{color}
\DeclareGraphicsExtensions{.pdf,.png,.jpg,.jpeg,.mps}
\usepackage{pgf}
\usepackage{epic,bez123}
\usepackage{floatflt}% package for floatingfigure environment
\usepackage{wrapfig}% package for wrapfigure environment

%-------------------------
%		ALGORITHMEN
%-------------------------
\usepackage{algorithm}
\usepackage{algpseudocode}
\renewcommand{\listalgorithmname}{List of algorithms}
\floatname{algorithm}{algorithm}

%-------------------------
%		SCHRIFTART
%-------------------------
% Entweder Latin Modern oder Times / Helvetica
\usepackage{lmodern} %Latin modern font
%\usepackage{mathptmx}  %Helvetica / Times New Roman fonts (2 lines)
%\usepackage[scaled=.92]{helvet} %Helvetica / Times New Roman fonts (2 lines)

%------------------------------------
%		KOPFZEILE / FUßZEILE
%------------------------------------
%	   ACHTUNG! Einige einstellungen werden in master.tex erneut verändert
\RequirePackage[automark,headsepline,footsepline]{scrpage2}
\pagestyle{scrheadings}
\renewcommand*{\pnumfont}{\upshape\sffamily}
\renewcommand*{\headfont}{\upshape\sffamily}
\renewcommand*{\footfont}{\upshape\sffamily}
\renewcommand{\chaptermarkformat}{}

\clearscrheadfoot

% using cfoot centers the header 
% use ofoot for outer side, especially with documentclass[twoside]
\cfoot[\rule{0pt}{\ht\strutbox+\dp\strutbox}\pagemark]{\rule{0pt}{\ht\strutbox+\dp\strutbox}\pagemark}

\ohead{\headmark}

%TODO: fix footer

%-----------------------------------
%		Fix Headheight
%-----------------------------------
\setlength{\headheight}{1.1\baselineskip}
