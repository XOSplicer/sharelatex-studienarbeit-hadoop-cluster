% !TEX root = ../master.tex

\chapter{Ansible Playbook Files for the System Set-Up}
\label{app:ansible}

\lstset{language=sh}
\begin{lstlisting}[caption={Ansible playbook that gets run to install the cluster system}, label={lst:ansibleall}]
---
# main playbook
- hosts: master
  roles:
    - role: master
      become: true
    - role: slave
      become: true

- hosts: slave
  roles:
    - role: slave
      become: true
\end{lstlisting}

\lstset{language=sh}
\begin{lstlisting}[caption={Common tasks in Ansible}, label={lst:ansiblecommon}]
---
# tasks file for common
- name: Set hostname
  hostname: name="{{ inventory_hostname }}"

- name: Set /etc/hosts
  template:
    src: "{{ role_path }}/templates/hosts.j2"
    dest: /etc/hosts
    owner: root
    group: root
    mode: 0644
    backup: yes

- name: Set authorized_keys for user ubuntu
  copy:
    src: "{{ role_path }}/files/authorized_keys"
    dest: /home/ubuntu/.ssh/authorized_keys
    owner: ubuntu
    group: ubuntu
    mode: 0600
    backup: yes

- name: Install common packages
  apt: pkg="{{ item }}" state=installed update_cache=true
  with_items:
    - "{{ pkg_pycurl }}"
    - "{{ pkg_curl }}"
    - "{{ pkg_unzip }}"
    - "{{ pkg_tar }}"
    - "{{ pkg_wget }}"

- name: Install ntp
  apt: pkg={{ pkg_ntp }} state=installed update_cache=true

- name: Activate ntp service
  service:
    name: ntp
    enabled: yes
    state: started

- name: Set interfaces file to load all interfaces in interfaces.d
  copy:
    src: "{{ role_path }}/files/interfaces"
    dest: /etc/network/interfaces
    owner: root
    group: root
    mode: 0644
    backup: yes

- name: Set internal network config
  template:
    src: "{{ role_path }}/templates/60-internal.cfg.j2"
    dest: /etc/network/interfaces.d/60-internal.cfg
    owner: root
    group: root
    mode: 0644
    backup: yes
  register: if_internal_restart

- name: Restart internal network interface
  shell: "ifdown {{ iface_internal }} && ifup {{ iface_internal }}"
  when: if_internal_restart | changed

- name: Download Ambari repository list
  get_url:
    url: "{{ ambari_repo_list }}"
    dest: /etc/apt/sources.list.d/ambari.list
    mode: 0644
    backup: yes

- name: Add Ambari repository key
  apt_key:
    keyserver: "{{ ambari_repo_keyserver }}"
    id: "{{ ambari_repo_keyid }}"

- name: Update apt cache
  apt:
    update_cache: yes

- name: Copy transparent-huge-pages service file
  copy:
    src: disable-transparent-hugepages
    dest: "/etc/init.d/disable-transparent-hugepages"
    mode: 0755

- name: Run service to disable transparent hugepages
  service:
    name: disable-transparent-hugepages
    state: started
    enabled: yes
\end{lstlisting}


\lstset{language=sh}
\begin{lstlisting}[caption={Tasks for the master role in Ansible}, label={lst:ansiblemaster}]
---
# tasks file for master
- name: Install ambari-server
  apt: pkg=ambari-server state=installed update_cache=true
  register: ambari_server_setup

- name: Set up ambari-server
  shell: "ambari-server setup --silent"
  when: ambari_server_setup | changed

- name: Activate ambari-server service
  service:
    name: ambari-server
    enabled: yes
    state: started
  ignore_errors: yes
\end{lstlisting}

\lstset{language=sh}
\begin{lstlisting}[caption={Tasks for the slave role in Ansible}, label={lst:ansibleslave}]
---
# tasks file for slave
- name: Mount Hadoop Data Volume
  mount:
    name: /hadoop
    src: "UUID={{ hadoop_vol_uuid }}"
    fstype: ext4
    state: mounted
    opts: rw,user,auto


- name: Install ambari-agent
  apt: pkg=ambari-agent state=installed update_cache=true

- name: Set ambari-agent.ini
  template:
    src: "{{ role_path }}/templates/ambari-agent.ini.j2"
    dest: /etc/ambari-agent/conf/ambari-agent.ini
    owner: root
    group: root
    mode: 0644
    backup: yes
  vars:
    master_hostname: "{{ hostvars[groups['master'][0]].inventory_hostname }}"
  register: ambari_agent_restart

- name: Restart ambari-agent service
  shell: "ambari-agent restart"
  when: ambari_agent_restart | changed

- name: Start ambari-agent service
  shell: "ambari-agent start"
  ignore_errors: yes
\end{lstlisting}