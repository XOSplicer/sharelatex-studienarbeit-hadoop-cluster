% !TEX root = ../master.tex
\chapter{Introduction}

Big Data is a phenomena in \ac{IT} that has emerged in the recent years.
The digitization across industries lead to rapidly growing amount of data generated
within each company.
The kinds of data generated are widespread 
and are influenced by the industry field, product range and used \ac{IT} systems 
of the individual firm.

From the business point of view, the available amount of data 
holds the opportunity for diverse analysis to gain insights on company-level scale.
\ac{BIA} is the field of study that describes technologies and strategies 
from initial gathering of data from various sources 
up to final decision making based on analysis results.    

\section{Big Data and Cluster Computing}


\section{Context}

\section{Motivation}

\section{Boundaries}

\section{Structure and Contribution}
\label{intro:structure}

%TODO: structure

As mentioned on page~\pageref{prenotes} this paper is part of a shared research project,
which is carried out by \textit{Jonas Balsfulland}, \textit{Felix Stegmaier} and \textit{Philipp Winter}.
Each author has a different scope for their work but the general topic of investigating the capabilities of an Hadoop cluster and implementing it for the \ac{DHBW} is the same.

%TODO: describe focuses and shared work 