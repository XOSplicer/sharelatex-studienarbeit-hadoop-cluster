% !TEX root = ../master.tex
\chapter{Introduction}
\label{chap:intro}

Big Data is a phenomena in \ac{IT} that has emerged in the recent years.
The digitization across industries lead to rapidly growing amount of data generated
within each company.
The kinds of data generated are widespread 
and are influenced by the industry field, product range and used \ac{IT} systems 
of the individual firm.

From the business point of view, the available amount of data 
holds the opportunity for diverse analysis to gain insights on company-level scale.
\ac{BIA} is the field of study that describes technologies and strategies 
from initial gathering of data from various sources 
up to final decision making based on analysis results.    

\section{Big Data and Cluster Computing}
\label{sec:intro:cluster}


\section{Context}
\label{sec:intro:context}

\section{Motivation}
\label{sec:intro:motivation}

\section{Boundaries}
\label{sec:intro:boundaries}

\section{Structure and Contribution}
\label{sec:intro:structure}

This paper is structured in multiple chapters that start at the basic understanding of the problem 
and lead to an final implementation of an solution.

\begin{enumerate}
    \item \emph{Chapter~\vref{chap:fund}} describes the fundamentals of the project. 
        It explains the general technical background of Hadoop 
        and how implementing an Hadoop cluster can benefit the \ac{DHBW}. 
    \item \emph{Chapter~\vref{chap:design}} focuses of the exploration of existing
        infrastructure and describes the creation of an architecture design 
        to be implemented in chapter~\ref{chap:impl}. 
        An execution plan is set up that can be followed to deploy the system.
    \item \emph{Chapter~\vref{chap:impl}} lays out how this plan is executed 
        and reflects on the outcomes of the implementation. 
    \item \emph{Chapter~\vref{chap:conc}} summarizes the paper 
        and discusses the outcome of the project critically.
        Further work that can succeed this project is listed.
\end{enumerate}

As mentioned on page~\pageref{chap:prenotes} this paper is part of a shared research project,
which is carried out by \emph{Jonas Balsfulland}, \emph{Felix Stegmaier} and \emph{Philipp Winter}.
Each author has a different scope for their work 
but the general topic of investigating the capabilities of an Hadoop cluster 
and implementing it for the \ac{DHBW} is the same.
The focuses are laid out as following:

\begin{enumerate}
	\item \emph{Felix Stegmaier} examines the existing infrastructure and assesses the feasibility of a deployment of \emph{Hadoop} as well as related technologies on top of it. Therefore, he focuses on creating an execution plan outlining the necessary steps, requirements and dependencies of the implementation.
	\item \emph{Jonas Balsfulland} aggregates an overview of existing applications utilizing \emph{Hadoop} as underlying data storage and/or task execution framework. This results in a use-of-potential analysis, considering also the complexity of integrating these solutions with a pre-existing Hadoop cluster.
	\item \emph{Philipp Winter} synthesizes use cases regarding \emph{\ac{BIA}}, especially considering their applicability to the \emph{Hadoop} platform and its related technologies. Furthermore, an overview of existing \emph{Data Science} programs offered by universities is to be researched. The use cases analysis in conjunction with promising module content can then be used to deduce a lecture outline. This lecture outline aims to introduce students into the topics of data science by offering a practical application of the Hadoop platform.
\end{enumerate}

The shared general topic leads to similarities in the basic aspects in each of the respective papers.
Therefore especially chapters 1 to 3 are created in cooperation 
and can contain sections which are literally or analogously equal without marking them as such.
