% !TEX root = ../master.tex
\chapter{Design and Methodology}
\label{chap:design}

\section{Analysis of Cluster Environment and Requirements}

TODO

\section{Research on Possibilities}

After the requirements are settled,
the possibilities of how Hadoop can actually be installed 
on the given infrastructure will be researched.

TODO

There are four ways to install Hadoop that are considered here:

\begin{itemize}
    \item A \emph{manual installation} of Hadoop 
        and every additional piece of software from the ecosystem
    \item \emph{\ac{CDH}} --  An open source software distribution by Cloudera including Hadoop
        and additional software from the ecosystem
    \item \emph{Ambari} -- An open source project by Apache that provides
        the capability to manage Hadoop Clusters and the software components running on it.
    \item \emph{\ac{HDP}} -- Hortonswork's ready-to-install distribution of Hadoop 
        with Ambari and additional software
\end{itemize}

TODO References

Since \ac{HDP} makes use of Ambari for installing the cluster and managing the installed software while it simply provides a pre-packaged version of the components, 
it can be preferred over using Ambari directly which would create the need to compile it from source.

In the coming sections each of those three resulting possibilities will be looked at in more detail.
For each of the options an evaluation will be made on whether it is 
feasible to deploy an Hadoop installation with the methods that they provide.
Therefore a simple test installation in an OpenStack environment is made.
For this purpose the OpenStack environment \emph{BWCloud}\footnote{https://www.bw-cloud.org/de/projekt} is used, an environment that is available for students at the \ac{DHBW} and other universities in Baden-Württemberg for educational purposes. It is generally an equivalent software to the actual cluster environment, however it is accessible from the internet, which makes it more comfortable to work with while performing the tests.

\subsection{Manual Installation of Hadoop}

TODO: Why is this the first choice

Tom White \autocite[][Appendix A]{white2015hadoop} describes how Hadoop can be manually installed on the machines in an cluster. 
Manually here means without the help of automated installers for a single machine or the complete cluster. 
Every aspect of the installation and configuration must be done \enquote{by hand}.
White gives step by step instruction on the process.

TODO describe where "distribution" comes from

For the installation two \acs{VM} are used, utilizing two virtual \ac{CPU} Cores and eight \ac{GB} \ac{RAM} and a 20 \ac{GB} storage drive. 
Ubuntu Server \footnote{https://www.ubuntu.com/server} 16.04.3 \ac{LTS} is used as the operating system. 
The machines are connected to the public internet and to each other using two network interfaces each.

The two machines are going to be connected together in an master-slave setup
where the master runs the \ac{YARN} resource manager and the \ac{HDFS} name node as well as an \ac{HDFS} data node and an \ac{YARN} client node.
The slaves runs only an \ac{HDFS} data node and an \ac{YARN} client node.

First the system of the master is configured. Therefore the following steps are taken.

\begin{enumerate}
    \item The system is updated and OpenJDK 1.8 is installed using\footnote{http://openjdk.java.net/} \texttt{apt}. 
    \item A new user \emph{hadoop} is created which will be used to run every service on the machine concerning Hadoop.
    \item For this user a new \ac{SSH} key is created which enables password-less access to the two machines for the user \emph{hadoop} if its public key is later added to the \texttt{authorized_key} file of the user on the hosts.
    \item The \texttt{/etc/hosts} file is adjusted, such that both hosts know how to reach each other on the local network by using their \acs{FQDN} which is resolved to an \ac{IP} address.
    \item The network interface to the internal network that is connected is configured to use the \ac{IP} address assigned to it by OpenStack.
    \item \ac{IP}v6 is disabled.
\end{enumerate}

The slave system uses mainly the same configuration but is altered slightly.
No new \ac{SSH} key is generated, but the public key of the previously
generated one is entered into the \texttt{authorized_key} file. 
Furthermore an additional 70 \ac{GB} hard drive is attached 
and mounted to store working data.

After the base system is configured, Hadoop can be installed on both machines
and can be configured to form a cluster.
Again, first the master host is configured.
Therefore the next steps describes in 
\autocite[][Appendix A]{white2015hadoop}
can be followed.

\begin{itemize}
    \item First Hadoop is downloaded from the Apache website\footnote{https://hadoop.apache.org/releases.html}. 
    At the moment of performing this tests, 
    version 2.7.4 is the latest stable release.
    The downloaded packed file is unzipped in the home directory of the \emph{hadoop} user.
    \item Inside the \texttt{.bashrc} file of the user the the variables \texttt{JAVA_HOME} is pointing to the OpenJDK installation,
    \texttt{HADOOP_HOME} is set to the folder that has just been crated and the \texttt{PATH} variable is configured to include the \textt{bin} directory within \texttt{HADOOP_HOME}.
    Each variable is exported to be available in the user shell.
    This makes Hadoop usable from the login shell of the user.
\end{itemize}

TODO hadoop config


TODO errors

TODO conclusion

TODO

\subsection{\acl{CDH}}

TODO

\subsection{\acl{HDP} with Apache Ambari}

TODO

\subsection{Decision on Distribution}

TODO

\section{Architecture Design}

TODO

\section{Execution Plan}

TODO
