% !TEX root = ../master.tex
\chapter{Implementation}
\label{chap:impl}

TODO short intro to sections to come

\section{Infrastructure Set-Up in OpenStack}

\subsection{Preparation}

TODO

    The OpenStack management web interface can be accesses at \urlinline{https://controller.c4.dhbw-mannheim.de/} from within the \ac{DHBW} network. 


TODO do the math on resources

TODO sec group considerations, which ports in out?




\subsection{Execution}

TODO

\subsection{Encountered Issues and Lessons Learned}

TODO Access only from dhbw on site net leads to difficult development conditions where author needs to be on site and is restricted by the locality

TODO Network connection unstable leading to more difficult environment with regular loss of connectivity to server

TODO Internal errors within openstack (temporary) where not enough resources could be allocated to get the requested VMs, later resolved


\section{System Set-Up with Ansible}

\subsection{Preparation}

\subsection{Execution}

\subsection{Encountered Issues and Lessons Learned}


\section{Hadoop Set-Up with Ambari}

\subsection{Preparation}

TODO explain and maybe print this particular playbook yes yes print it with explainations

\subsection{Execution}

\subsection{Encountered Issues and Lessons Learned}

\section{System Tests}

\section{Conclusion}

TODO dhbw cloud is unreliable af

TODO process is way too hard for and too long for students to do in a lecture without depper understanding of sysadmin  



TODO HELP ME IM STREGGELING WITH EXISTENCE

\urlinline{https://docs.hortonworks.com/HDPDocuments/Ambari-2.6.1.5/bk_ambari-installation/bk_ambari-installation.pdf}


TODO ref exec plan

TODO Tools will be Apache Spark, Storm, Hive und Pig

TODO openstack

    ATTENTION: Network connection: sometimes due to a bug in openstack first only outer network connection then attach inner connection and reboot at least in bwclound, in dhbw works. 
    , note down the eth port that is newly created for internal connection.
    


TODO Ansible
see \urlinline{https://github.com/XOSplicer/studienarbeit-hadoop-cluster-ansible}
     (in implementation explain how the ansible part works). 
     
     should create also new user which is allowed to be accessed via SSH and  password so that hadoop can be used from this account

TODO Ambari

    TODO ,  when promted to select nodes use manual registaration: reason dont upload ssh private key for root access (maybe in chpt 4).

TODO Tests

\urlinline{https://gist.github.com/ace-subido/0a9b219b2348921f6a87/3141000d2cbb0f78b967b75304908f4289aa8f01}

\urlinline{https://www.ripublication.com/ijaer18/ijaerv13n6_166.pdf}
\urlinline{http://sortbenchmark.org/YahooHadoop.pdf}
\urlinline{http://citeseerx.ist.psu.edu/viewdoc/download?doi=10.1.1.178.1187&rep=rep1&type=pdf}

\urlinline{https://hadoop.apache.org/docs/r1.0.4/api/org/apache/hadoop/examples/terasort/TeraGen.html}\\urlinline{https://www.systutorials.com/3235/hadoop-terasort-benchmark/} (also mentioned in \autocite[][]{white2015hadoop})


TODO

TODO maintainability
