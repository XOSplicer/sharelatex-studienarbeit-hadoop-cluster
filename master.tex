%
%   Prof. Dr. Julian Reichwald
%   auf Basis einer Vorlage von Prof. Dr. Jörg Baumgart
%   DHBW Mannheim
%
%	ACHTUNG: Für das Erstellen des Literaturverzeichnisses wird das modernere Paket biblatex
%			 in Kombination mit biber verwendet -- nicht mehr das ältere BibTex!
% 			 Bitte stellen Sie ggf. Ihre TeX-Umgebung
% 			 entsprechend ein (z.B. TeXStudio: Einstellungen --> Erzeugen --> Standard Bibliographieprogramm: biber)
%

\documentclass[
	12pt,
	BCOR=5mm,
	DIV=12,
	headinclude=on,
	footinclude=off,
	parskip=half,
	bibliography=totoc,
	listof=entryprefix,
	toc=listof,
	number=noenddot,
	plainfootsepline]{scrreprt}

%	Konfigurationsdatei einbinden
% !TEX root = ../master.tex

%-------------------
% 		HYPERREF
%-------------------

\usepackage[hidelinks=true]{hyperref}
 % hidelinks=true verhindert rote Ränder bei Links im Dokument.

% Zwei eigene Befehle zum Setzen von Autor und Titel. Ausserdem werden die PDF-Informationen richtig gesetzt.
\newcommand{\titel}[1]{\def\dertitel{#1}\hypersetup{pdftitle={#1}}}
\newcommand{\autor}[1]{\def\derautor{#1}\hypersetup{pdfauthor={#1}}}

%-----------------------------------
%		SCHRIFT UND ENCODING
%-----------------------------------
\usepackage[T1]{fontenc}
\usepackage[utf8]{inputenc}

\usepackage{setspace}
%\onehalfspacing
%TODO use this spacing

%---------------------------
%		BERECHNUNGEN
%---------------------------
\usepackage{calc} % Used for extra space below footsepline

%---------------------------------
%		SPRACHEINSTELLUNGEN
%---------------------------------
% Voreinstellungen für Deutsch und Englisch. Die nicht verwendete Sprache ist auszukommentieren.
% DEUTSCH
%\usepackage[ngerman]{babel}
%\usepackage[german=quotes]{csquotes}

%ENGLISCH
\usepackage[english]{babel}
\usepackage{csquotes} % Richtiges Setzen der Anführungszeichen mit \enquote{}

%----------------------------
%		BIBLIOGRAFIE
%----------------------------
% Voreinstellungen für Fußnotenzitate (Autor-Jahr), IEEE-Standard, Alphabetic-Stil und Havard-Stil. Die nicht verwendeten Stile müssen auskommentiert werden

% \usepackage[backend=biber, autocite=footnote, style=authoryear, dashed=false]{biblatex}		% Fußnotenzitate
\usepackage[backend=biber, autocite=inline, style=ieee]{biblatex}							% IEEE-Stil
% \usepackage[backend=biber, autocite=inline, style=alphabetic]{biblatex}					% Alphabetic-Stil
% \usepackage[backend=biber, autocite=inline, style=authoryear, dashed=false]{biblatex}		% Harvard-Stil

% Fußnotenzitate mit YYYY-MM-DD in Bibliographie
% \usepackage[backend=biber, autocite=footnote, style=authoryear, dashed=false, urldate=edtf, date=edtf, seconds=true]{biblatex}

% Zum Zählen der Fußnoten über Kapitel hinaus
\usepackage{chngcntr}
\counterwithout{footnote}{chapter}

\DefineBibliographyStrings{ngerman}{  %Change u.a. to et al. (german only!)
	andothers = {{et\,al\adddot}},
}

\setlength{\bibparsep}{\parskip}		%add some space between biblatex entries in the bibliography
\addbibresource{adds/bibliography.bib}	%Add file bibliography.bib as biblatex resource

%----------------------
%		ACRONYME
%----------------------
%%%
%%% WICHTIG: Installieren Sie das neueste Acronyms-Paket!!!
%%%
\makeatletter
\usepackage[printonlyused]{acronym}
\@ifpackagelater{acronym}{2015/03/20}
  {%
    \renewcommand*{\aclabelfont}[1]{\textbf{\textsf{\acsfont{#1}}}}
  }%
  {%
  }%
\makeatother

%--------------------
%		GLOSSAR
%--------------------
%\usepackage[toc]{glossaries}					% für Seitenreferenzen im Glossar
\usepackage[toc, nonumberlist]{glossaries}		% ohne Seitenreferenzen im Glossar

%---------------------
%		LISTINGS
%---------------------
\usepackage{listings}
% Listings formatieren
%\renewcommand{\lstlistlistingname}{Quelltextverzeichnis}
\lstset{numbers=left,
	numberstyle=\tiny,
	captionpos=b,
	breaklines=true,
	basicstyle=\linespread{0.8}\ttfamily\small}

%-------------------------------
%		ZUSÄTZLICHE PAKETE
%-------------------------------
\usepackage{lipsum}				% Blindtext
\setlipsumdefault{1-4}
\usepackage[pdftex]{graphicx} 			% verschiene Bildformate einbinden
\usepackage{pdfpages}		% PDF einbinden
\usepackage{varioref} 	% schönere Referenzen über \vref{}
\usepackage{caption}			% schönere Überschriften
\usepackage{booktabs}			% bessere Tabs
\usepackage{array}
\newcolumntype{P}[1]{>{\raggedright\arraybackslash}p{#1}}

%--------------------------
%		Tikz diagram library
%--------------------------
\usepackage{tikz}
\usetikzlibrary{arrows,decorations.pathmorphing,backgrounds,fit,positioning,shapes.symbols,chains}

%--------------------------
%		TpX used packages
%--------------------------
\usepackage{color}
\DeclareGraphicsExtensions{.pdf,.png,.jpg,.jpeg,.mps}
\usepackage{pgf}
\usepackage{epic,bez123}
\usepackage{floatflt}% package for floatingfigure environment
\usepackage{wrapfig}% package for wrapfigure environment

%-------------------------
%		ALGORITHMEN
%-------------------------
\usepackage{algorithm}
\usepackage{algpseudocode}
\renewcommand{\listalgorithmname}{List of algorithms}
\floatname{algorithm}{algorithm}

%-------------------------
%		SCHRIFTART
%-------------------------
% Entweder Latin Modern oder Times / Helvetica
\usepackage{lmodern} %Latin modern font
%\usepackage{mathptmx}  %Helvetica / Times New Roman fonts (2 lines)
%\usepackage[scaled=.92]{helvet} %Helvetica / Times New Roman fonts (2 lines)

%------------------------------------
%		KOPFZEILE / FUßZEILE
%------------------------------------
%	   ACHTUNG! Einige einstellungen werden in master.tex erneut verändert
\RequirePackage[automark,headsepline,footsepline]{scrpage2}
\pagestyle{scrheadings}
\renewcommand*{\pnumfont}{\upshape\sffamily}
\renewcommand*{\headfont}{\upshape\sffamily}
\renewcommand*{\footfont}{\upshape\sffamily}
\renewcommand{\chaptermarkformat}{}

\clearscrheadfoot

% using cfoot centers the header 
% use ofoot for outer side, especially with documentclass[twoside]
\cfoot[\rule{0pt}{\ht\strutbox+\dp\strutbox}\pagemark]{\rule{0pt}{\ht\strutbox+\dp\strutbox}\pagemark}

\ohead{\headmark}

%TODO: fix footer

%-----------------------------------
%		Fix Headheight
%-----------------------------------
\setlength{\headheight}{1.1\baselineskip}

%\input{adds/glossary}

%\makeglossaries

\begin{document}

%----------------------------------------
% Titel und Autor der Arbeit hier angeben
%----------------------------------------
\titel{Hadoop Cluster}
\autor{Felix Stegmaier}

% !TEX root = ../master.tex
\begin{titlepage}

%\begin{minipage}{\textwidth}
%		\vspace{-2cm}
%		\noindent  \hfill   \includegraphics[height=2cm, keepaspectratio]{img/dhbw_logo.png}
%		
%		\vspace{1cm}
%\end{minipage}
\begin{center}
        \v
		\includegraphics[width=7cm, keepaspectratio]{img/dhbw_logo.png}
\end{center}

\vspace{1em}

\sffamily
\begin{center}
	\textsf{\large{}Cooperative State University\\[1.5mm] Stuttgart}\\[2em] \vspace{1cm}
	\textsf{\textbf{\Large{}Student Research Project}}\\[3mm] \vspace{1cm}
	\textsf{\textbf{\dertitel}} \\[1.5cm]	\vspace{1cm}
	\textsf{\textbf{\Large{}Applied Computer Science}\\[3mm]}

	\vspace{2cm}
	%\textsf{\Large{Sperrvermerk}}
\vfill

\begin{minipage}{\textwidth}

\begin{tabbing}
	Scientific Supervisor: \hspace{1.85cm}\=\kill
	Author: \> \derautor \\[1.5mm]
	Student Number: \> 6079153\\[1.5mm]
	Course: \> TINF15A\\[1.5mm]
	Group Name: \> C27BFB49\\[1.5mm]
	Group Members: \> Jonas Balsfulland\\
	\> Felix Stegmaier\\
	\> Philipp Winter\\[1.5mm]
%	Head of Department: \> Prof. Dr. Dirk Reichardt, Prof. Dr. Carmen Winter\\[1.5mm]
	Scientific Supervisor: \> Prof. Dr. Carmen Winter \\[1.5mm]
	Date of Submission: \> 2018-06-04\\
\end{tabbing}
\end{minipage}

\end{center}

\end{titlepage}


\pagenumbering{Roman} % Römische Seitennummerierung
\normalfont

%----------------------------------------
% Verzeichnisse - nicht benötige Verzeichnisse bitte auskommentieren / löschen.
%----------------------------------------

%   Sperrvermerk
%% !TEX root = ../master.tex
\chapter*{Confidentiality Notice}
\thispagestyle{scrheadings}
This work as a whole or in part may
not be disclosed to persons outside of the exam and evaluation process,
provided that no other approval is given by the training institution or the author.
\cleardoublepage


% Ehrenwörtliche Erklärung
% !TEX root = ../master.tex
\clearpage
\chapter*{Declaration of Authorship}
\thispagestyle{empty}

% Wird die folgende Zeile auskommentiert, erscheint die ehrenwörtliche
% Erklärung im Inhaltsverzeichnis.

% \addcontentsline{toc}{chapter}{Ehrenwörtliche Erklärung}

I hereby declare:

\begin{itemize}
	\item that this paper with the title \textit{\dertitel} is my own work and
	\item that I have not used any other sources or assistance than declared here.
	\item that I have not submitted the paper as a whole or in part for an degree at any university
		or institution before.
	\item that I have not published this paper before.
	\item Furthermore I confirm, that the presented electronical version of this paper
		is identical to the printed version.
\end{itemize}
I am aware, that an incorrect declaration will be followed by legal measures.

\vspace{3cm}
City, Date
\vspace{3em}

\rule{6cm}{0.4pt}\\
\derautor


%	Kurzfassung
% !TEX root = ../master.tex
\chapter*{Abstract}

\begingroup
  \begin{table}[h!]
    \setlength\tabcolsep{0pt}
    \begin{tabular}{p{3.5cm}p{10.0cm}}
      Title & \dertitel \\
      Author: & \derautor \\
    \end{tabular}
  \end{table}
\endgroup

\hspace{2cm}

The ever-increasing amount of data available in the current digital age 
poses major challenges for corporations, 
a phenomenon widely known as \emph{Big Data}. 
In order to gain insight from this data,
one needs to have information systems in place which 
provide a platform to analyze it in an efficient manner.

The computing power of single computers is limited by both physical 
as well as economic constraints,
hence a multitude of machines can be connected to form \emph{computing clusters}
that are capable o.

As the computing power of each computer is limited by both physical 
as well as economic constraints, 
the scale of nowadays' Big Data needs to be handled by
a multitude of connected machines that therefore form \emph{computing clusters}.
Such cluster environments can scale with the increasing amount of data managed by it. 
The various nodes of each cluster have to be organized,
e.g. by assigning tasks to a particular machine or sending data to places where it is needed. 
This work is performed by so-called \emph{cluster managers}.

A popular cluster manager for Big Data clusters in the industry is the \emph{Hadoop} framework. 
Hadoop offers distributed storage for arbitrary-sized collections of data 
by combining the individual computing and storage resources of commodity hardware. 
Furthermore, it provides programming interfaces for interacting with and analyzing the stored data.

This research project is embedded into a larger body of investigation 
regarding the aptitude of Hadoop as \acs{BIA} information system 
for use at the \acf{DHBW} in both research and administration. 
Thus, this work specifically deals with the following topics:

\begin{itemize}
    \item Examination of existing infrastructure
    \item Deployment of an Hadoop cluster on-top of it. 
    This also includes the exploration of possibilities for automatic configuration 
    and installation of such a cluster.
\end{itemize}

As an outcome of this project, 
a detailed execution plan to perform said deployment is given.
Furthermore the deployment is rolled out to the given environment. 

%TODO: Update after project




% Vorangehende Notitzen
% !TEX root = ../master.tex
\clearpage
\chapter*{Preliminary Notes}
\thispagestyle{empty}


\paragraph{Group Work}
This paper has been created as a cooperation between multiple Authors: 
Jonas Balsfulland, Felix Stegmaier and Philipp Winter.
Therefore the resulting three papers have common content.
Especially parts of chapter 1 to 3 may be shared literally or analogously between the papers.

%TODO which chapers exactly
%TODO where are differences in focus


\paragraph{Gender Neutrality}
This paper is written in gender neutral language.
Any person is addressed using the singular \textit{they}.
For example, instead of 
``The user changes \textit{his} account settings.''
this paper writes 
``The user changes \textit{their} account settings.''


\paragraph{Trademarks}
Designations from third-party vendors are used in this paper.
These designations may be trademarks or registered trademarks.
Wherever the author is aware of such trademark, 
the designation will be written in initial caps.




%	Inhaltsverzeichnis
\begingroup
\renewcommand*{\chapterpagestyle}{empty}
\pagestyle{empty}
\tableofcontents
\clearpage
\endgroup

\pagestyle{scrheadings}

%	Abbildungsverzeichnis
\listoffigures

%	Tabellenverzeichnis
\listoftables

%	Listingsverzeichnis
\lstlistoflistings

% 	Algorithmenverzeichnis
\listofalgorithms

% 	Abkürzungsverzeichnis (siehe Datei acronyms.tex!)
% !TEX root = ../master.tex
\clearpage
\chapter*{List of Acronyms}
\addcontentsline{toc}{chapter}{List of Acronyms}

%Verwendung:
%		\ac{Abk.}   --> fügt die Abkürzung ein, beim ersten Aufruf wird zusätzlich automatisch die ausgeschriebene Version davor eingefügt bzw. in einer Fußnote (hierfür muss in header.tex \usepackage[printonlyused,footnote]{acronym} stehen) dargestellt
%		\acs{Abk.}   -->  fügt die Abkürzung ein
%		\acf{Abk.}   --> fügt die Abkürzung UND die Erklärung ein
%		\acl{Abk.}   --> fügt nur die Erklärung ein
%		\acp{Abk.}  --> gibt Plural aus (angefügtes 's'); das zusätzliche 'p' funktioniert auch bei obigen Befehlen
%	siehe auch: http://golatex.de/wiki/%5Cacronym

\begin{acronym}[LongExample]
\setlength{\itemsep}{-\parsep}
    \acro{ACL}{Access Control List}
	\acro{API}{Application Programming Interface}
	\acro{BDA}{Big Data Analytics}
	\acro{BDSG}{Bundesdatenschutzgesetz (Federal Data Protection Act)}
	\acro{BFS}{Big File System}
	\acro{BIA}[BI\&A]{Business Intelligence \& Analytics}
	\acro{CDH}{Cloudera Distribution Including Apache Hadoop}
	\acro{CPU}{Central Processing Unit}
	\acro{CSV}{Comma Separated Values}
	\acro{DBMS}{Database Management System}
	\acro{DHBW}{Duale Hochschule Baden-Württemberg (Cooperative State Univerity Baden-Württemberg)}
	\acro{DM}{Data Mining}
	\acro{DNS}{Domain Name System}
	\acro{DWH}{Data Warehouse}
	\acro{ECC}{Error-Correcting Code}
	\acro{ETL}{Extract Transform Load}
	\acro{EU}{European Union}
	\acro{FQDN}{Fully Qualified Domain Name}
	\acro{GDPR}{General Data Protection Regulation}
	\acro{GB}{gigabyte}
	\acro{HDFS}{Hadoop Distributed File System}
	\acro{HDP}{Hortonworks Data Platform}
	\acro{HSchulDSV}{Hochschuldatenschutzverordnung (data protection regulations for institutions of higher education)}
	\acro{IEEE}{Institute of Electrical and Electronics Engineers}
	\acro{IoT}{Internet of Things}
	\acro{IP}{Internet Protocol}
	\acro{IT}{Information Technology}
	\acro{JDK}{Java Development Kit}
	\acro{LTS}{Long Term Support}
	\acro{NCDC}{National Climatic Data Center}
	\acro{NTP}{Network Time Protocol}
	\acro{OLAP}{Online Analytical Processing}
	\acro{OLTP}{Online Transaction Processing}
	\acro{VM}{Virtual Machine}
	\acro{RAM}{Random Access Memory}
	\acro{RDD}{Resilient Distributed Dataset}
    \acro{RDBMS}{Relational Database Management System}
    \acro{SLA}{Service Level Agreement}
    \acro{SQL}{Structured Query Language}
	\acro{SSH}{Secure Shell}
	\acro{TB}{terabyte}
	\acro{THP}{Transparent Hugh Pages}
	\acro{YAML}{YAML Ain’t Markup Language}
	\acro{YARN}{Yet Another Resource Negotiator}
\end{acronym}


%----------------------------------------
% Start des Textteils der Arbeit
%----------------------------------------
\clearpage
\ihead{\chaptername~\thechapter} % Neue Header-Definition
\pagenumbering{arabic}  % Arabische Seitenzahlen

%Einzelne Kapitel können hier eingefügt werden.
%Es ist vorgesehen alle Kapitel als eigene Dateien
%% !TEX root = ../master.tex
\chapter{Introduction}
%\lipsum
\autocite{TW15}
%% !TEX root = ../master.tex
\chapter{Fundamentals}
\label{chap:fund}

TODO

\section{Big Data, \acf{BIA} and \acf{BDA}}

Chen et.~al characterize \ac{BIA} as \blockcquote[p.~1166]{chen2012business}{the techniques, technologies, systems, practices, methodologies, and applications that analyze critical business data to help an enterprise better understand its business and market and make timely business decisions}. Its goals and methodology closely resembles that of the emerging field of \ac{BDA} \autocite[][p.~1166]{chen2012business}. 
According to Chen et.~al and their involvement in the \ac{BIA} space from the end of the last century to the 2010s three major evolutionary phases of the sector may be differentiated, with \ac{BDA} characterizing the evolution in the last stage \autocite[][p.~1168 \psqq]{chen2012business}. In order to provide an overview regarding the development of the context area, the different stages will be looked at and briefly characterized hereafter.

In the first evolutionary phase, identified as \textbf{\ac{BIA} 1.0}, the desire for the storage and analysis of business data emerged. Companies started to utilize already existing \acp{RDBMS} to store mostly structured data collected from various legacy systems such as mainframes. This phase includes the usage of statistical methods to generate and validate new hypotheses regarding stored information, widely known as \ac{DM}. In this regard the main emphasis of \ac{BIA} departments was to design, deploy and maintain data warehouses with optimized \ac{ETL} cycles in order to regularly feed new data into the system using batch processing. The stored information may then be analyzed using \ac{OLAP} systems and are usually visualized with dashboards and graphs to allow a better insight for the end-users.
Most of the currently deployed analytic platforms, including standard software by common software vendors such as Microsoft or Oracle, are based on the technologies developed in the \ac{BIA} 1.0 phase. Although these developments already started in the 1970s and accelerated in the 1980, there is still ongoing research and active feature implementation until today. \autocite[Cmp.][p.~1166 \psq]{chen2012business}

As the Web 2.0 became popular in the early 2000s, an unknown variety of data sources emerged as well. The goal of the \textbf{\ac{BIA} 2.0} ignited by these events is to harvest the new opportunities for analytical processing created by this additional data. The collection of statistics on an individual customer basis allowed to generate a better understanding of customer's needs and thus offer tailored goods and services to a company's clients.
Moreover, the invention of social media platforms led to a massive influx of publicly available user-generated content. The analysis of such unstructured data posed the main challenge to traditional applications, focused on structured data with well-defined schemes. Although some techniques and opportunities addressed by \ac{BIA} 2.0 have been incorporated into commercial offerings, there still exists a large room for future growth. Especially the techniques of text mining, social network and spatial-temporal analysis are still under active academic investigation. \autocite[Cmp.][p.~1167 \psq]{chen2012business}

Finally, the most interesting phase in the context of \ac{BDA} is that of \textbf{\ac{BIA} 3.0}. Sparked by an exponentially increasing volume of data due to mobile and \ac{IoT} devices, that \enquote{traditional} systems are not capable of dealing with, this term represents a new research area currently in development. The enormous desire of enterprises to gain meaningful insight from the \blockcquote[p.~1168]{chen2012business}{massive, location-aware, person-centered and context-relevant data} in a so-called Web 3.0 environment is leading both IT vendors and academia to develop tailor-fit solutions. However, as of today,  no comprehensive standard software is commercially available yet, and neither are academic courses that educate the workforce of tomorrow. \autocite[Cmp.][p.~1168]{chen2012business}

Rajasekar et. al. establishes a understanding on the definitions and tools of \ac{BDA}, enriched by offering a comparison with traditional \acp{DBMS} regarding size and type of the underlying input data \autocite[Cmp.][p.~80]{rajasekar2015survey}. Accordingly, \ac{BDA} is concerned with the analysis of the \enquote{massive amount of un-, semi- and structured data too large to be handled by traditional DBMS}. The property of being too large is therefore strongly dependent on the precise \ac{DBMS} in use \autocite[Cmp.][p.~80]{rajasekar2015survey}, and the threshold level might change in the future as technological advancements enable legacy software to process larger data quicker. However, additional processing power created by technological advancements is unlikely to keep up with the exponential growth of incoming data. \autocite[Cmp.][p.~80\psq]{rajasekar2015survey}

\begin{table}[hbt]
	\begin{tabular}{l|ll}
	  \textbf{Characteristic} & \textbf{Traditional \acp{RDBMS}} & \textbf{\acf{BDA}} \\[0.5em]
	  \hline
	  Data Size & Gigabytes & Petabytes \\
	  Access Type & Interactive and Batch & Batch only \\
	  Update Frequency & Continuously Read and Write & Write once, Read continuously \\
	  Schema Type & Static & Dynamic \\
	  Data Integrity & High & Low \\
	  Scaling Behavior & Nonlinear & Linear \\
	\end{tabular}
	\caption{Comparison of typical \acl{RDBMS} and \acl{BDA} properties. Reprinted with adaptions from \cite[p.~80]{rajasekar2015survey}}
	\label{fig-dbms-vs-bda}
\end{table}


In order to compare the most important properties of traditional \acp{RDBMS} with those of
\ac{BDA} methods, i.e. \emph{MapReduce}, the authors propose a table similar to
\autoref{fig-dbms-vs-bda}. The table highlights the flexibility that is connected with \ac{BDA},
achieved by breaking up the structure of traditional system and limiting the interactive
possibilities for the prospective end users, which are traded against a much higher maximum
throughput. \autocite[Cmp.][p.~80]{rajasekar2015survey}
Furthermore, it is industrially accepted that the differences between the two data analysis
paradigms can be grouped into five dimensions, the so-called \emph{5V's}, as displayed in
\autoref{fig-5vs} \autocite[Cmp.][p.~80]{rajasekar2015survey},\autocite[Cmp.][p.~3\psq]{bhosale2014review}.

\begin{figure}[hbt]
  {\centering\includegraphics[width=0.75\textwidth]{{resources/big-data-parameters}.png}\par}
  \caption{The \emph{5V's} of \acl{BDA}. Reprinted from \autocite[][p.~81]{rajasekar2015survey}}
  \label{fig-5vs}
\end{figure}

As soon as data increases complexity in at least some of these dimensions, it qualifies for special treatment through \ac{BDA} systems. Although this is a common use case, as outlined in the next section, only few software solutions are available in the open market to combat these challenges. \autocite[Cmp.][p.~1182 \psqq]{chen2012business},\autocite[Cmp.][p.~80]{rajasekar2015survey}


\section{Hadoop}

TODO copy paste

\subsection{Development of Hadoop}

TODO

\subsection{Hadoop Technical Explanation}

TODO

\subsection{Hadoop at the DHBW}

TODO

\section{Legal Boundaries}

When a \ac{BDA} system is set up, it handles data in diverse forms from many sources.
Depending on the kind of data there are different legal restriction 
to the conditions of collection, storage and processing thereof.
The main restrictive laws that are applicable for the \ac{DHBW} are the Euroean \ac{GDPR} and the \ac{HSchulDSV} Baden-Würtemberg.

Table \ref{fig-legal-data-kinds} describes which restrictions regulations apply to which type of data from which source. 
An example for the kind of data is given.

\begin{table}[hbt]
\resizebox{\textwidth}{!}{%
	\begin{tabular}{l|lll}
	  & \textbf{Internal} & \textbf{Open Source} &  \textbf{Closed Source} \\[0.5em]
	  \hline
	  \textbf{Personal} & \acs{GDPR} \& \acs{HSchulDSV} & N/A & \acs{GDPR} \& license \\
	  & e.g. student data & N/A & e.g. advertising contacts\\[0.5em]
	  \textbf{Anonymized} & possibly \acs{GDPR} \& \acs{HSchulDSV} & possibly \acs{GDPR} & possibly \acs{GDPR} \& license\\
	  & e.g. lecture statistics & e.g. social media graphs & e.g. advertising statistics\\[0.5em]
	  \textbf{Non-Personal} & unrestricted & unrestricted & restricted by license \\
	  & e.g. server monitoring & e.g. weather data & e.g. bought datasets \\
	\end{tabular}%
	}
	\caption{Kinds of data and how their use might be restricted by legislation or licenses.}
	\label{fig-legal-data-kinds}
\end{table}





The \acf{HSchulDSV} in Baden-Würtemberg \autocite[§1, §12][]{bw2012hcchuldsv}

Hochschuldatenschutzverordnung (HSchulDSV BW) §1, §12 \autocite{bw2012hcchuldsv}
Data protection regulations for institution of higher education in Baden-Württemberg

European \ac{GDPR} \autocite{eu2016gdpr}

The \ac{GDPR} Portal \autocite{trunomi2018gdpr} summarizes the key changes that were introduced:
\begin{itemize}
    \item \emph{Penalties} may be imposed for organizations that breach the \ac{GDPR}.
    \item \emph{Consent} from users must be requested to collection, storage and usage of personal data. The consent may be revoked.
    \item \emph{Breach Notifications} are mandatory to inform users if data was exposed.
    \item \emph{Right to Access} for every user to their own personal data.
    \item \emph{Data Portability}, i.e. the requested data should be available to the user in a common format.
    \item \emph{Right to be Forgotten}, i.e. the right to have personal data deleted permanently upon request.
    \item \emph{Privacy by Design} for new data processing systems.
\end{itemize}

TODO conclusion

%\input{content/03Problem}
%\input{content/04Analysis}
%\input{content/05Design}
%\input{content/06Implementation}
%\input{content/07Conclusions}
\input{adds/anleitung}
\input{content/99kapitel}

%	Literaturverzeichnis
\ihead{} % Neue Header-Definition
\printbibliography

%\printglossaries

%----------------------------------------
% Anhang
% Hier können alle Anhänge als einzelne Datei eingefügt werden.
% Die Anhänge werden gesondert alphabethisch benannt (A Anhang 1, B Anhang 2, usw)
%----------------------------------------
\appendix
\ihead{\appendixname~\thechapter} % Neue Header-Definition

% !TEX root = ../master.tex

%\chapter{Appendix A}



\end{document}
